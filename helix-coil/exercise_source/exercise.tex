\documentclass[a4paper]{article}
\usepackage{titling}
\usepackage{authblk}
\usepackage{fancyhdr}
\usepackage{hyperref}
\usepackage{rsc}
\usepackage{siunitx}
\usepackage{graphicx}
\usepackage{listings}
\usepackage{amsmath}
\usepackage{color}

\definecolor{dkgreen}{rgb}{0,0.6,0}
\definecolor{gray}{rgb}{0.5,0.5,0.5}
\definecolor{mauve}{rgb}{0.58,0,0.82}


\lstset{frame=tb,
  language=Python,
  aboveskip=3mm,
  belowskip=3mm,
  showstringspaces=false,
  columns=flexible,
  basicstyle={\ttfamily},
  numbers=none,
  numberstyle=\tiny\color{gray},
  keywordstyle=\color{blue},
  commentstyle=\color{dkgreen},
  stringstyle=\color{mauve},
  breaklines=true,
  breakatwhitespace=true,
  tabsize=3
}
\DeclareSIUnit\Fahrenheit{\degree F}
\setlength{\parindent}{0pt}
\parskip 1.5ex

\title{Problem Solving Exercise 1: Protein helix to coil transition}
\author[1]{Dr Benjamin J. Morgan}
\author[1,2]{Dr Andrew R. McCluskey}
\affil[1]{Department of Chemistry, University of Bath, email: b.j.morgan@bath.ac.uk}
\affil[2]{Diamond Light Source, email: andrew.mccluskey@diamond.ac.uk}
\setcounter{Maxaffil}{0}
\renewcommand\Affilfont{\itshape\small}

\pagestyle{fancy}
\fancyhf{}
\rhead{CH40208}
\lhead{\thetitle}
\rfoot{\thepage}

\begin{document}
\maketitle

This exercise involves the modeling of cooperative protein unfolding using statistical thermodynamics. 
You should use the information in the provided handout to carry out the folliwng tasks. 

\begin{table}[h]
    \centering
    \caption{Transition energies for each of the steps in a $n=4$ amino acid system.}
    \label{tab:energy}
    \begin{tabular}{c | c | c}
        Transition & Symbol & Energy/\si{\joule} \\
        \hline
        $0 \rightarrow 1$ & $q_1$ & \num{1.1e-19} \\
        $0 \rightarrow 2$ & $q_2$ & \num{2.0e-19} \\
        $0 \rightarrow 3$ & $q_3$ & \num{2.9e-19} \\
        $0 \rightarrow 4$ & $q_4$ & \num{3.7e-19} \\
    \end{tabular}
\end{table}

\begin{enumerate}
    \item Write a function, that given an arbitrary number of amino acids in a randomly coiled conformation, $i$, and total amino acids, $n$, can determine the number of possible permutations $C(n, i)$. \\ Note, that the function \texttt{factorial} from the \texttt{scipy.special} module is capbale of returning an array of factorials from a NumPy array. 
    \item Table~\ref{tab:energy} gives the energy for the transitions of a fully helical protein chain, of length 4, to having 1, 2, 3, and 4 randomly coiled amino acids. The partition coefficient associated with the fully helical structure is $q_0 = 1.21$. 
    \begin{enumerate}
        \item Using Equations~3 and 4 from the handout, determine the value of the total paritition function, $q$, for this system at a temperature of \SI{273.15}{\kelvin}.
        \item Determine the total partition function at \SIlist{200;400;600;800;1000}{\kelvin}, and plot the variance as a function of temperature.  
    \end{enumerate}
    \item The partition function of this process can be generalised, initially by applying a non-cooperative model, create a function to evaluate the non-cooperative model for an arbitrary number of amino acids and plot how the result of this varies in the temperature series mentioned above for proteins with $4$, $20$, and $100$ amino acids, assume a value for $\gamma$ of \SI{1e-19}{\joule}.
    \item The zipper model for protein unfolding is a form of cooperative model that is outlined in Equation 10.
    \begin{enumerate}
        \item Write three functions to determine; the partition function for the cooperative model, the fraction of proteins that would be found with $i$ randomly coiled amino acids, and the mean number of randomly coiled amino acids.
        \item Plot the variation of $p_i$ as a function of $i$ for a protein consisting of 20 amino acids, where $sigma = $\num{5e-3} and $s$ is 1.5, 1, and 0.82. 
        \item Determine the mean $i$ for each of the above stability parameters. 
    \end{enumerate}
\end{enumerate}

\end{document}
